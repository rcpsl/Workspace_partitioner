\section{Framework}

\subsection{LiDAR Image Function}

In general, LiDAR image dependence on quadrotor position is nonlinear even with workspace fixed.
It only becomes affine when quadrotor is constrained in a small region such that obstacle boundary
intersect each laser from the onboard LiDAR does not change with quadrotor position.
We formally define obstacles detected by a LiDAR, and claim that LiDAR image is a piecewise affine function in the following:

\begin{definition}
    A LiDAR configuration is a list of obstacle boundaries detected by a LiDAR, and the list is sorted by laser directions.
\end{definition}

\begin{theorem}
    For a given workspace with polygonal obstacles, if laser directions are fixed, 
    and LiDAR detection range is large enough such that each laser intersects an obstacle boundary, 
    then LiDAR image is a piecewise affine function of quadrotor position:
    \begin{equation}
        \label{eq:image_func}
        d_t = P_j [x_{p_t} \quad y_{p_t}]^\intercal + q_j, \quad \forall j \in {1, ..., M} 
    \end{equation}
    where $M$ is number of regions identified by LiDAR configuration.
\end{theorem}    

To prove this theorem, we propose an algorithm that partition workspace based on LiDAR configuration.

\begin{proof}
    First, partition free space by segments that have one end be obstacle vertices, one end on obstacle boundaries, 
    and oriented along laser directions.
    Such partition segments can be found by assuming rays start from obstacle vertices,
    in all laser directions except those not point to free space.
    Intersections between these rays with the first obstacles encountered would be the other endpoint of partition segments.
    Next, all intersections between partition segments can be found by a sweep algorithm. 
    {\color{blue} More detail about sweep if necessary.}
    Consider a graph consists of all partition segments and intersections,
    then each region that cannot be further divided can be found by graph search.
    Each of these regions corresponds to a unique LiDAR configuration.
    
    Assume quadrotor with LiDAR onboard is constrained in an arbitrary one of these regions, 
    then obstacle boundary intersects with each laser is determined due to the uniqueness of LiDAR configuration for each region.
    In other words, for each laser direction $\theta^i \in {1, ..., N}$, there exists $0 \le k^i \le 1$ such that:
    \begin{equation}
        \label{eq: intersection1}
        x^i = x_{p_t} + k^i R cos\theta^i, \qquad y^i = y_{p_t} + k^i R sin\theta^i, \\
    \end{equation} 
    \begin{equation}
        \label{eq: intersection2}
        x_{obs}^s \le x^i \le x_{obs}^l, \qquad y_{obs}^s \le y^i \le y_{obs}^l
    \end{equation} 
    where constant $R$ is maximum detection range of LiDAR.
    Constants $x_{obs}^s$, $x_{obs}^l$, $y_{obs}^s$, $y_{obs}^l$ parameterize the obstacle boundary in LiDAR configuration corresponds
    to laser direction $\theta^i$.
    Since laser directions do not rotate when quadrotor turn around,  
    $\theta^i$ are constants and hence $x^i$, $y^i$ are affine functions of quadrotor position $(x_{p_t}$, $y_{p_t})$.
    Therefore, LiDAR image $d_t$ that consists of linear terms of $x^i$, $y^i$ is an affine function of quadrotor position.
\end{proof}

As a special case, consider obstacles are all rectangles. 
A laser in direction $\theta^i$, where $i \in {1, ..., N}$, intersects an vertical obstacle 
defined by lower end $(x_{obs}, y_{obs}^s)$ and upper end $(x_{obs}, y_{obs}^l)$, 
constraints ~\eqref{eq: intersection1}, ~\eqref{eq: intersection2} can be simplified as:
\begin{equation}
    \label{eq:vertical}
    x^i = x_{obs}, \qquad y^i = y_{p_t} + (x_{obs} - x_{p_t}) tan \theta^i
\end{equation}
Similarly, for a horizontal obstacle boundary with left end $(x_{obs}^s, y_{obs})$ and right end $(x_{obs}^l, y_{obs})$,
constraints ~\eqref{eq: intersection1}, ~\eqref{eq: intersection2} can be written as:
\begin{equation} 
    \label{eq:horizontal}
    x^i = x_{p_t} + (y_{obs} - y_{p_t}) cot \theta^i, \qquad y^i = y_{obs}
\end{equation}


\begin{theorem}
    Given workspace with polygonal obstacles, running time of partitioning workspace based on LiDAR configuration is
    $\mathcal{O}(n\log{}n + I\log{}n)$, where $n$ is number of partition line segments 
    and $I$ is number of intersection points of segments.
    \begin{proof}
        {\color{blue} This is complexity of sweep algorithm.} 
    \end{proof}
\end{theorem}    

{\color{blue} In practice have numerical errors.}



\subsection{State Machine}


For a given ReLU assignment, safety verification involves both feedback control loop and environment 
is equivalent to solve a feasibility problem with constraints 
~\eqref{eq:dyn}, ~\eqref{eq:ulimit}, ~\eqref{eq:image}, ~\eqref{eq:nn_linear}, ~\eqref{eq:image_func}.
However, this optimization problem is non-convex due to the piecewise linearity of LiDAR image function.
Instead of solving the problem as a whole, we consider transition feasibility between 
each pair of regions, each corresponds to a fixed LiDAR configuration. 
Thus, we add constraints that quadrotor is located in region $R_j$ currently and moves to region $R_{j^\prime}$,
$\forall j \neq j^\prime \in {1, ..., M}$:
\begin{equation}
    \label{eq:region}
    E_j[x_{p_t} \quad y_{p_t}]^\intercal <= f_j, \qquad E_{j^\prime}[x_{p_{t+1}} \quad y_{p_{t+1}}]^\intercal <= f_{j^\prime}
\end{equation}
Based on transition feasibility between all regions, a state machine is available and can be used to verify
LTL specifications.

{\color{blue} How to use state machine to verify LTL?}
{\color{blue} Any limit on LTL formula?}

{\color{blue} Talk about abstraction refinement.}

