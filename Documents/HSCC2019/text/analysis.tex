\section{Algorithm Analysis}

\subsection{Hardness of Building State Machine}

\begin{theorem}
    Given a NN-controlled system 
    $\mathcal{S} = \langle (A, B), \overline{u}, \overline{x}, \mathcal{L}, \mathcal{N} \rangle$ 
    and a pair of regions from a partitioned workspace, 
    the problem of determining whether constraints
    ~\eqref{eq:dyn}, ~\eqref{eq:ulimit}, ~\eqref{eq:nn_nonlinear}, ~\eqref{eq:image_func}
    can be satisfied simultaneously is NP-complete.
\end{theorem}    

\begin{proof}
    {\color{blue} 
    (The problem is NP) A certificate can be simply checked by forwarding through NN.

    (NP-hard) It can be reduced to the problem of determining whether NN is satisfiable for linear range of input and output, 
    which has been proved to be NP-complete by the Reluplex paper.
    }
\end{proof}


\begin{theorem}
    Given a NN-controlled system 
    $\mathcal{S} = \langle (A, B), \overline{u}, \overline{x}, \mathcal{L}, \mathcal{N} \rangle$ 
    and a partitioned workspace,
    the problem of determining whether there exists two regions such that constraints
    ~\eqref{eq:dyn}, ~\eqref{eq:ulimit}, ~\eqref{eq:nn_nonlinear}, ~\eqref{eq:image_func}
    can be satisfied simultaneously is $\Delta_2^P = P^{NP}$.
\end{theorem}    

\begin{proof}
    {\color{blue} 
    The problem can be solved in polynomial time by a Turing machine with an oracle for NP-complete problem, 
    i.e. polynomially many calls of NP-complete problem.
    }
\end{proof}    



{\color{blue} Are above two theorems enough or need complexity of building state machine, which is not a decision problem?}


\subsection{Soundness and Completeness}
{\color{blue} Sound but incomplete.}

