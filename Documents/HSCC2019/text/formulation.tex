\section{Problem Formulation}

\subsection{Dynamics and Workspace}

We assume that the dynamics of a quadrotor is described by a discrete-time linear system of the form:

\begin{equation}
    \label{eq:dyn}    
    x_{t+1} = A x_{t} + B u_{t} 
\end{equation}

\begin{equation}
    \label{eq:ulimit}    
    \norm{x_t} \le \overline{x}, \qquad \norm{u_t} \le \overline{u}, \quad \forall t \in \N
\end{equation}
where $x_t \in \mathcal{X} \subseteq \R^{n}$ is the state of quadrotor at time $t \in \N$, 
$u_t \in \mathcal{U} \subseteq \R^{m}$ is the quadrotor input, $\norm{\cdot}$ denotes the infinity norm,
and $\overline{u}$ and $\overline{x}$ are bounds on the input and state variables. 
The matrices $A$ and $B$ represent the quadrotor dynamics and have appropriate dimensions. 
For a quadrotor with nonlinear dynamics that is either differentially flat or feedback linearizable, 
the state space model~\eqref{eq:dyn} corresponds to its feedback linearized dynamics.

{\color{blue} May need to emphasize quadrotor does not rotate when make a turn, which is briefly mentioned below.}

We consider quadrotor in a workspace $\mathcal{W} \subset \R^{w}$ where $w$ can be $2$ or $3$, 
corresponding,  respectively, to a $2$-dimensional or $3$-dimensional workspace. 
Assume that quadrotor must avoid a set of \emph{obstacles} $\mathcal{O} = \{\mathcal{O}_1, \ldots, \mathcal{O}_o\}$, 
with $\mathcal{O}_i \subset \R^w$ is assumed to be polygon.
We call obstacle boundary for both hyperplanes bound obstacles and workspace, and similarly for obstacle vertices. 

\subsection{Neural Network}
We consider neural network as a feedback controller that makes decision based on current states of quadrotor
and observation of environment.
In other words, neural network can be considered as a nonlinear function of LiDAR image:
\begin{equation}    
    \label{eq:nn_nonlinear}    
    u_t = f_{NN}(d_t)
\end{equation}    

\begin{equation}
    \label{eq:image}
    d_t = [x^1-x_{p_t},..., x^N-x_{p_t}, y^1-y_{p_t},..., y^N-y_{p_t}, x_{goal}, y_{goal}]^\intercal
\end{equation}
where $x_{p_t}$, $y_{p_t}$ are position states in $x_t$,
$N$ is the number of lasers from onboard LiDAR of quadrotor. 
Note that instead of directly using distances to obstacles, LiDAR image $d_t$ consists of $(x^i, y^i)$, 
$i\in\{1,\ldots,N\}$, which are coordinates of intersections between each laser $i$ and the obstacle it intersects.
Computing intersection coordinates by distances is straightforward provided all laser directions are fixed, 
which is satisfied by the real motion pattern of a quadrotor. $x_{goal}$, $y_{goal}$ correspond to the goal 
that quadrotor tries to reach.


We consider piecewise linear layers of neural network.
{\color{blue} Talk about SMC framework: SAT assigns ReLUs.}
After fixing ReLU phases, the nonlinear function~\eqref{eq:nn_nonlinear} is equivalent to the following linear function 
and constraints on domain of linearity:
\begin{equation}
    \label{eq:nn_linear}  
    u_t =  G_b d_t + h_b, \qquad Q_b d_t \le c_b
\end{equation}
where $G_b$, $h_b$, $Q_b$, $c_b$ depend on ReLU phases and their dimensions are determined by the size of neural network and input image.


\subsection{Linear Temporal Logic}

{\color{blue} LTL background.}



