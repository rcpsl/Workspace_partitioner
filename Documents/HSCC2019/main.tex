\documentclass[sigconf,anonymous]{acmart}

\usepackage{booktabs} % For formal tables

% XS: add command
\newcommand{\norm}[1]{|\!|#1|\!|}
\newcommand{\N}{{\mathbb{N}}}
\newcommand{\R}{{\mathbb{R}}}
\newcommand{\W}{{\mathcal{W}}}



\usepackage{color}
\renewcommand{\r}[1]{{{{\color{red}#1}}}}
\renewcommand{\b}[1]{{{{\color{blue}#1}}}}


% Copyright
%\setcopyright{none}
%\setcopyright{acmcopyright}
%\setcopyright{acmlicensed}
\setcopyright{rightsretained}
%\setcopyright{usgov}
%\setcopyright{usgovmixed}
%\setcopyright{cagov}
%\setcopyright{cagovmixed}


% DOI
\acmDOI{XXXX/XXXX}

% ISBN
\acmISBN{XXX-XXXX-XX-XXX/XX/XX}

%Conference
\acmConference[HSCC'19]{ACM International Conference on 
Hybrid Systems: Computation and Control}{April 2019}{Montreal, Canada}
\acmYear{2019}
\copyrightyear{2019}


%\acmArticle{X}
%\acmPrice{XX}

% These commands are optional
%\acmBooktitle{Transactions of the ACM Woodstock conference}
%\editor{Jennifer B. Sartor}
%\editor{Theo D'Hondt}
%\editor{Wolfgang De Meuter}

\settopmatter{printacmref=false}



\begin{document}
\title{System-Level Verification of Neural Network Controlled Autonomous Systems}

%\titlenote{Produces the permission block, and copyright information}
%\subtitle{Extended Abstract}
%\subtitlenote{The full version of the author's guide is available as
%  \texttt{acmart.pdf} document}

%
%\author{Ben Trovato}
%\authornote{Dr.~Trovato insisted his name be first.}
%\orcid{1234-5678-9012}
%\affiliation{%
%  \institution{Institute for Clarity in Documentation}
%  \streetaddress{P.O. Box 1212}
%  \city{Dublin}
%  \state{Ohio}
%  \postcode{43017-6221}
%}
%\email{trovato@corporation.com}
%
%\author{G.K.M. Tobin}
%\authornote{The secretary disavows any knowledge of this author's actions.}
%\affiliation{%
%  \institution{Institute for Clarity in Documentation}
%  \streetaddress{P.O. Box 1212}
%  \city{Dublin}
%  \state{Ohio}
%  \postcode{43017-6221}
%}
%\email{webmaster@marysville-ohio.com}
%
%\author{Lars Th{\o}rv{\"a}ld}
%\authornote{This author is the
%  one who did all the really hard work.}
%\affiliation{%
%  \institution{The Th{\o}rv{\"a}ld Group}
%  \streetaddress{1 Th{\o}rv{\"a}ld Circle}
%  \city{Hekla}
%  \country{Iceland}}
%\email{larst@affiliation.org}
%
%\author{Valerie B\'eranger}
%\affiliation{%
%  \institution{Inria Paris-Rocquencourt}
%  \city{Rocquencourt}
%  \country{France}
%}
%\author{Aparna Patel}
%\affiliation{%
% \institution{Rajiv Gandhi University}
% \streetaddress{Rono-Hills}
% \city{Doimukh}
% \state{Arunachal Pradesh}
% \country{India}}
%\author{Huifen Chan}
%\affiliation{%
%  \institution{Tsinghua University}
%  \streetaddress{30 Shuangqing Rd}
%  \city{Haidian Qu}
%  \state{Beijing Shi}
%  \country{China}
%}
%
%\author{Charles Palmer}
%\affiliation{%
%  \institution{Palmer Research Laboratories}
%  \streetaddress{8600 Datapoint Drive}
%  \city{San Antonio}
%  \state{Texas}
%  \postcode{78229}}
%\email{cpalmer@prl.com}
%
%\author{John Smith}
%\affiliation{\institution{The Th{\o}rv{\"a}ld Group}}
%\email{jsmith@affiliation.org}
%
%\author{Julius P.~Kumquat}
%\affiliation{\institution{The Kumquat Consortium}}
%\email{jpkumquat@consortium.net}
%
%% The default list of authors is too long for headers.
%\renewcommand{\shortauthors}{B. Trovato et al.}


\begin{abstract}
In this paper, we consider the formal verification problem of an autonomous robot equipped with a LiDAR scanner and a Neural Network (NN) that processes the LiDAR images to produce control actions. Given a workspace that is characterized by a set of obstacles and a set of regions of interest, where both the obstacles and the regions are polyhedra, we show that the LiDAR imaging function (that maps the robot position to the LiDAR image) is a piecewise \r{affine} function. Based on this observation, we develop a \r{polynomial}-time algorithm that partitions the workspace into regions such that the LiDAR imaging is \r{affine}. Given this workspace partitioning, a discrete-time and linear dynamics of the robot, and a pre-trained NN controller with Rectifier Linear Unit (ReLU) nonlinearity, our objective is to compute a finite transition abstraction of the NN-controlled system. Motivated by the NP-hardness of analyzing neural networks with ReLU nonlinearities, we develop a Satisfiability Modulo Convex (SMC) Programming algorithm that utilizes a Boolean satisfiability solver and a convex programming solver and decomposes the problem into smaller subproblems. At each iteration, the  Boolean satisfiability solver searches for a candidate assignment for the different ReLU phases while completely abstracting the robot dynamics. Convex programming is then used to check the feasibility of the proposed ReLU phases against the dynamic and imagining constraints, or generate succinct explanations for their infeasibility to reduce the search space. 
%This process needs to be executed while taking into considerations each combination of regions in the partitioned workspace.  To harness the exponential growth due to the large number of partitioned regions, our framework utilizes an abstraction refinement process in which coarse abstractions based on a smaller number of laser beams are first considered and the refined if deemed necessary by the SMC-based procedure. 
Finally, given the computed finite transition abstraction along with a system-level property $\varphi$ captured by Linear-Temporal Logic (LTL), our framework can verify if the NN-controlled system satisfies $\varphi$. Numerical results show the effectiveness of our framework in proving the safety of a NN-controlled robot with \r{XX} continuous states and a neural network consisting of \r{XX} neurons.
\end{abstract}

%
% The code below should be generated by the tool at
% http://dl.acm.org/ccs.cfm
% Please copy and paste the code instead of the example below.
%
%\begin{CCSXML}
%<ccs2012>
% <concept>
%  <concept_id>10010520.10010553.10010562</concept_id>
%  <concept_desc>Computer systems organization~Embedded systems</concept_desc>
%  <concept_significance>500</concept_significance>
% </concept>
% <concept>
%  <concept_id>10010520.10010575.10010755</concept_id>
%  <concept_desc>Computer systems organization~Redundancy</concept_desc>
%  <concept_significance>300</concept_significance>
% </concept>
% <concept>
%  <concept_id>10010520.10010553.10010554</concept_id>
%  <concept_desc>Computer systems organization~Robotics</concept_desc>
%  <concept_significance>100</concept_significance>
% </concept>
% <concept>
%  <concept_id>10003033.10003083.10003095</concept_id>
%  <concept_desc>Networks~Network reliability</concept_desc>
%  <concept_significance>100</concept_significance>
% </concept>
%</ccs2012>
%\end{CCSXML}
%
%\ccsdesc[500]{Computer systems organization~Embedded systems}
%\ccsdesc[300]{Computer systems organization~Redundancy}
%\ccsdesc{Computer systems organization~Robotics}
%\ccsdesc[100]{Networks~Network reliability}


\keywords{Formal verification, Machine Learning, Satisfiability Solver}


\maketitle

%\input{text/samplebody-conf}
\section{Introduction}
From simple logical constructs to complex deep neural network models, Artificial Intelligence (AI)-agents are increasingly controlling physical/mechanical systems. Self-driving cars, drones, and smart cities are just examples of such systems to name a few. However, regardless of the explosion in the use of AI within a multitude of cyber-physical systems (CPS) domains, the safety and reliability of these AI-enabled CPS is still an under-studied problem. It is then unsurprising the failure of these AI-controlled CPS in several, safety-critical, situations leading to human fatalities~\cite{AccidentWiki}. 

%Recent polls show that the societal rejection of these technologies increases significantly with every failure~\cite{VergePoll} leading to more than 75\% of  Americans being too afraid to ride in a self-driving vehicle~\cite{VergePoll} and 64\% afraid from sharing the road with autonomous cars~\cite{VergePoll2}. \textbf{Such safety and reliability concerns, if not proactively addressed, will pose a significant societal barrier of adopting these technologies permanently~\cite{MITPoll,ForbesPoll}}. 


Motivated by the urgency to study safety, reliability, and potential problems that can rise and impact the society by the deployment of AI-enabled systems in the real world, several works in the literature focused on the problem of designing deep neural networks models that are robust to the so-called adversarial examples~\cite{ferdowsi2018robust,everitt2018agi,charikar2017learning,steinhardt2017certified,munoz2017towards,paudice2018label,ruan2018global}. Unfortunately, these techniques focus mainly on the robustness of the learning algorithm with respect to data outliers without providing guarantees in terms of safety and reliability of the decisions taken by the neural network model. To circumvent this drawback, and motivated by the wealth of adversarial example generation approaches for neural network, recent works focused on two main techniques namely (i) testing of neural networks, (ii) falsification (semi-formal verification) of neural networks, and (iii) formal verification of neural networks.

Representatives of the first class, namely testing of neural networks, are the work reported in~\cite{pei2017deepxplore,tian2017deeptest,wicker2018feature,YouchengTesting2018,LeiDeepGauge2018,Wang2018Testing,LeiDeepMutation2018,srisakaokul2018multiple,MengshiDeepRoad2018,YouchengConcolic2018} in which the neural network is treated as a white box, and test cases are generated to maximize different coverage criteria. Such coverage criteria include neuron coverage, condition/decision coverage, and multi-granularity testing criteria. On the one hand, testing do not formally guarantee that a neural network satisfy a formal functionality, on the other hand, maximizing test coverage give system designers confidence that the networks are reasonably free from defect. %Nevertheless, the major drawback of the neural network testing is that focuses entirely on the neural network as a component without taking into consideration the system-level 


Two major drawbacks of current testing schemes of neural networks. First, they focus entirely on testing the neural network as a component without taking into consideration the effect of its decisions on the entire system behavior. Second, testing schemes do not utilize a formal notion of component-level or system-level safety specifications. This motivated researchers to focus on falsification of autonomous systems that include machine learning components~\cite{dreossi2017compositional,tuncali2018simulation,zhang2018two}. In such falsification frameworks, the objective is to generate corner test cases that will lead the whole system to violate a system-level specification. To that end, advanced 3D models and image environments are used to bridge the gap between the virtual world and the real world. By parametrizing the input to these 3D models (e.g., position of objects, position of light sources, intensity of light sources) and sampling the parameter space in a fashion that maximizes the falsification of the safety property, falsification frameworks can simulate several test cases until a counterexample is found~\cite{dreossi2017compositional,tuncali2018simulation,zhang2018two}.


While testing and falsification frameworks are powerful tools to find corner cases in which the neural network or the neural network enabled system will fail, they lack the rigor promised by formal verification methods. Therefore, several researchers pointed to the urgent need of using formal methods to verify the behavior of neural networks and neural network enabled system~\cite{kurd2003establishing,seshia2016towards,seshia2018formal,leikeAIsafety2017,leofante2018automated,scheibler2015towards}. As a result, several works have been reported in the last few years attempting to apply formal verification techniques to neural network models. 
%The work in this area can be classified into two categories namely (i) component-level verification and (ii) system-level verification. 

However, applying formal verification to neural network models comes with its unique challenges. First and foremost is the lack of widely-accepted, precise, mathematical specifications capturing the correct behavior of a neural network. Therefore, recent works focused entirely on verifying neural networks against simple input-output specifications~\cite{katz2017reluplex,ehlers2017formal,bunel2018unified,ruan2018reachability,dutta2017output,pulina2010abstraction}. Such input-output techniques compute a guaranteed range for the output of a deep neural network given a set of inputs represented as a convex polyhedron.  

To that end, several algorithm that takes advantage of the piecewise linear nature of the Rectified Linear Unit (ReLU) activation functions (one of the most famous nonlinear activation functions in deep neural networks) have been proposed recently. For example, by using binary variables to encode piecewise linear functions, the constraints of ReLU functions are encoded as a Mixed-Integer Linear Programming (MILP). Combining output specifications that are expressed in terms of Linear Programming (LP), the verification problem for output set eventually turns to the feasibility problem of MILP~\cite{dutta2018output,tjeng2017verifying}. 

Using off-the-shelf MILP solvers does not lead to a scalable approach to handle neural networks with hundreds and thousands of neurons~\cite{ehlers2017formal}. To circumvent this problem, several MILP-like solvers targeted toward the neural network verification problem are recently proposed. For example, the work reported in~\cite{katz2017reluplex} proposed a modified Simplex algorithm (originally used to solve linear programs) to take into account ReLU nonlinearities as well. Similarly, the work reported in~\cite{ehlers2017formal} combines a Boolean satisfiability solving along with a linear over-approximation of piecewise linear functions to verify ReLU neural networks against convex specifications. Other techniques that exploit specific geometric structures of the specifications are also proposed~\cite{gehr2018ai,xiang2017reachable}. A thorough survey on different algorithms for verification of neural networks against input-output range specifications can be found in~\cite{xiang2017survey} and the references within.

%For example, by taking advantage of the piecewise linear nature of ReLU activation functions, the output set computation can be formulated as operations of polytopes if the input set is given in the form of unions of polytopes. The computation process involves standard polytope operations, such as intersection and projection, and all of these can be computed by employing sophisticated computational geometry tools. However, the number of polytopes involved in the computation process increases exponentially with the number of neurons in its worst case performance which makes the method not scalable to neural networks with a large number of neurons.


%The use of binary variables to encode piecewise linear functions is standard in optimization. In [87], the constraints of ReLU functions are encoded as a Mixed-Integer Linear Programming (MILP). Combining output specifications that are expressed in terms of Linear Programming (LP), the verification problem for output set eventually turns to the feasibility problem of MILP. It is well known that MILP is an NP-hard problem and in [35, 36], the authors elucidate significant efforts for solving MILP problems efficiently to make the approach scalable. Their methods combine MILP solvers with a local search yielding a more efficient solver for range estimation problems of ReLU neural networks than several other approaches. 

%Recently, a verification engine for ReLU neural networks called AI$^2$ was proposed in [47]. In their approach, the authors abstract perturbed inputs and safety specifications as zonotopes, and reason about their behavior using operations for zonotopes. The framework AI$^2$ is capable of handling neural networks of realistic size, and, in particular, their approach has had success dealing with convolutional neural networks. In another work, a software tool, called Sherlock, was developed based on the MILP verification approaches [38]. This LP-based framework combines satisfiability (SAT) solving and linear over-approximation of piecewise linear functions in order to verify ReLU neural networks against convex specifications.


%In [76], an algorithm, that stems from the Simplex Algorithm for linear functions, for ReLU functions is proposed. Due to the piecewise linear feature of ReLU functions, each node is divided into two nodes. Thus, in their formulation, each node consists of a forward-facing and backward-facing node. If the ReLU semantics are not satisfied, two additional update functions are given to fix the mismatching pairs. Thus, the search process is similar to the Simplex Algorithm that pivots and updates the basic and non-basic variables with the addition of a fixing process for ReLU activation pairs.






% First and foremost is the lack of widely-accepted, precise, mathematical specifications capturing the correct behavior of the neural network. Instead of verifying AI-agents against \textbf{component-level specifications}, we focus instead on verifying the whole CPS against \textbf{system-level specifications} like a ``self-driving car should avoid hitting a pedestrian crossing the street''. On the one hand, system-level specifications are interpretable by end users of CPS and can directly address their safety and reliability concerns. On the other hand, verifying system-level specifications necessitates novel decision procedures capable of simultaneously reasoning about the physical and cyber components in CPS, a computationally daunting problem. 


%\paragraph{Robust Learning and Adversarial examples. } New advances in AI systems have created an urgency to study safety, reliability, and potential problems that can rise and impact the society by the deployment of AI-based systems in the real world. Amodei et al. have explored concrete problems in AI safety including avoiding side effects and reward hacking, scalable supervision, safe exploration, and distributional shift~\cite{amodei2016concrete}. Malicious use of artificial intelligence~\cite{brundage2018malicious}, robustness of deep learning, and AI techniques for safety (with regards to the so-called adversarial examples) have been addressed by various groups~\cite{ferdowsi2018robust,everitt2018agi,charikar2017learning,steinhardt2017certified,munoz2017towards,paudice2018label,ruan2018global}. Unfortunately, these techniques focus mainly on the robustness of the learning algorithm with respect to data outliers without providing guarantees in terms of safety and reliability of the decisions taken by the AI-agent.


%\paragraph{Formal Verification of AI-Systems. }
%Due to lack of rigorous mathematical analysis of AI-agents, several researchers pointed to the urgent need of using formal methods to verify the behavior of AI-agents~\cite{kurd2003establishing,seshia2016towards,seshia2018formal,leikeAIsafety2017,leofante2018automated,scheibler2015towards}.  

%Several works have been reported in the last three years attempting to apply formal verification techniques to machine learning components and in particular neural networks. The work in this area can be classified into two categories namely (i) component-level and (ii) system-level verification. 
%
%In the first class, component-level verification, recent works focused on verifying neural networks against input-output specifications~\cite{katz2017reluplex,ehlers2017formal,bunel2018unified,ruan2018reachability,dutta2018output,pulina2010abstraction}. Such input-output techniques compute a guaranteed range for the output of a deep neural network given a set of inputs represented as a convex polyhedron. Unfortunately, these range properties do not capture the safety and reliability of the AI-agent, especially when operated in dynamic and changing environments.


%To circumvent the drawback of using simple input-output range specifications and reason directly about system safety, the second class of recent works in the literature focused on finding corner-cases that lead to the violation of system safety specifications. Unfortunately, recent works focused entirely on testing and semi-formal verification (e.g., falsification)~\cite{pei2017deepxplore,tian2017deeptest,wicker2018feature,YouchengTesting2018,LeiDeepGauge2018,Wang2018Testing,LeiDeepMutation2018,srisakaokul2018multiple,MengshiDeepRoad2018,YouchengConcolic2018,dreossi2017compositional}. While testing and falsification are useful in finding some corner-cases for which the system may fail, they lack the rigor promised by formal methods.
%In this work, we gain inspiration from these previous work to study formal verification of AI-controlled agents against system-level specifications. 





Unfortunately, the input-output range properties, studied so far in the literature, are simplistic and fails to capture the safety and reliability of the whole system. Therefore, in this paper, we focus instead on the problem of formal verification of a neural network controlled robot against system-level safety specifications. In particular, we consider the problem in which a robot utilizes a LiDAR scanner to sense its environment. The LiDAR image is then processed by a neural network controller which computes the control actions based on the current LiDAR image. Such scenario is common in the literature of behavioral cloning and imitation control in which the neural network is trained to imitate the actions of experts who manually controlled the robot~\r{\cite{??}}. With the objective to verify the safety of this robot, we develop a framework that can take into account the robot continuous dynamics, the workspace configuration, the LiDAR imaging, and the neural network, and compute the set of initial states for the robot that is guaranteed to produce robot trajectories that satisfy the safety specification.


To carry out the prescribed formal verification problem, we need a mathematical model that captures the LiDAR imaging process. This is the process that generates the LiDAR images based on the robot pose along with the workspace objects.  Therefore, the first contribution of this paper is to show that, under mild assumptions on the workspace, the robot dynamics, and the pre-processing of the LiDAR images, the mathematical model of the LiDAR imaging process enjoys favorable mathematical structure that renders it amenable to formal verification. In particular, we show that the LiDAR imaging process can be modeled as a piecewise affine function and we develop a polynomial-time algorithm that can partition the workspace into regions according to the imaging function.

%Unfortunately, the number of partitioned regions depend on the number of laser rays used by the LiDAR scanner. 

Given the partitioned workspace along with a pre-trained neural network and the robot dynamics, we compute a finite state abstraction of the closed loop dynamics. Such abstraction will be used later to verify the safety specifications. Similar to previous works in the literature, we strict our focus to neural networks with Rectified Linear Unit (ReLU) nonlinearities and we develop a Satisfiability Modulo Convex (SMC) programming algorithm that uses a combination of a Boolean satisfiability solver and a convex programming solver to iteratively reason about the neural network nonlinearity along with the dynamics and the imaging constraints. At each iteration, the boolean satisfiability solver searches for a candidate assignment for the ReLU phases while ignoring the neural network weights, the robot dynamics, and the LiDAR imaging. The convex programming solver is then used to check the feasibility of the proposed ReLU phase assignment against the neural network weights, the robot dynamics, and the LiDAR imaging. If the ReLU phase assignment is deemed infeasible, then the SMC solver will generate succinct explanations for their infeasibility to reduce the search space. 

Unfortunately, the SMC algorithm needs to be executed while taking into considerations each combination of regions in the partitioned workspace. Since the number of regions grows exponentially with the number of laser rays used by the LiDAR scanner, our framework utilizes an abstraction refinement process in which coarse abstractions based on a smaller number of laser rays are first considered and the refined if deemed necessary by the SMC-based procedure. Our simulation results show that our framework is capable of proving the safety of an autonomous robot with \r{XX} continuous states controlled by a neural network that consists of \r{XX} neurons. To summarize, the contributions of this paper can be summarized as follows:\\
\textbf{1-} A framework for formally proving safety properties of autonomous robots controlled by neural network processing LiDAR images.\\
\textbf{2-} An algorithm for computing a piece-wise affine representation of the LiDAR imaging process. \\
\textbf{3-} An SMC-based algorithm combined with an abstraction refinement process for computing finite abstractions of the neural network controlled autonomous robot.

%The remainder of this paper is organized as follows. In Section~\ref{??} we give .. In Section~\ref{??}..










\section{Problem Formulation}


\subsection{Dynamics and Workspace}

We assume that the dynamics of a robot is described by a discrete-time linear system of the form:

\begin{equation}
    \label{eq:dyn}    
    x_{t+1} = A x_{t} + B u_{t} 
\end{equation}

\begin{equation}
    \label{eq:ulimit}    
    \norm{x_t} \le \overline{x}, \qquad \norm{u_t} \le \overline{u}, \quad \forall t \in \N
\end{equation}
where $x_t \in \mathcal{X} \subseteq \R^{n}$ is the state of robot at time $t \in \N$, 
$u_t \in \mathcal{U} \subseteq \R^{m}$ is the robot input, $\norm{\cdot}$ denotes the infinity norm,
and $\overline{u}$ and $\overline{x}$ are bounds on the input and state variables. 
The matrices $A$ and $B$ represent the robot dynamics and have appropriate dimensions. 
For a robot with nonlinear dynamics that is either differentially flat or feedback linearizable, 
the state space model~\eqref{eq:dyn} corresponds to its feedback linearized dynamics.

We consider robot in a workspace $\W \subset \R^{w}$ where $w$ can be $2$ or $3$, 
corresponding,  respectively, to a $2$-dimensional or $3$-dimensional workspace. 
Assume that robot must avoid a set of \emph{obstacles} $\mathcal{O} = \{\mathcal{O}_1, \ldots, \mathcal{O}_o\}$, 
with $\mathcal{O}_i \subset \R^w$ is assumed to be polyhedron.
We denote obstacle boundaries for both boundaries of obstacles and workspace, and similarly for obstacle vertices. 



\subsection{LiDAR Image}
We consider an autonomous robot that detects environment by an onboard LiDAR scanner, 
which measures distances to obstacles in a set of $N$ directions.
We assume the detecting directions are fixed and do not rotate with robot.
Consider distance measurement in a particular direction $\theta^i$ is $r^i, i\in\{1,\ldots,N\}$,
it is straightforward to compute its components along $x$ and $y$ axes:
\begin{equation}
    \label{eq:distance}
    x^i = r^i cos \theta, \qquad y^i = r^i sin \theta
\end{equation}

Besides the relative distances between robot and obstacles, we take position of robot $(x_{p_t}, y_{p_t})$
at time $t \in \N$ into account.
Therefore, LiDAR image $d_t$ can be expressed as following:
\begin{equation}
    \label{eq:image}
    d_t = [x^1-x_{p_t},..., x^N-x_{p_t}, y^1-y_{p_t},..., y^N-y_{p_t}]^\intercal
    %d_t = [x^1-x_{p_t},..., x^N-x_{p_t}, y^1-y_{p_t},..., y^N-y_{p_t}, x_{goal}, y_{goal}]^\intercal
\end{equation}
%where $x_{goal}$, $y_{goal}$ correspond to the goal that robot tries to reach.




\subsection{Neural Network}

A neural network is comprised of multiple layers, with multiple neurons in each layer. 
In particular, we consider a NN architecture consists of fully connected layers, which have each neuron connect
to all neurons in the preceding layer.
The connection is mathematically characterized by weights, which are determined during training phase.
In order to represent nonlinear functions, an activation function is applied to output of all neurons except in the last layer.
One of the most common choice of activation function is Rectified Linear Unit (ReLU)~\cite{Hinton2010ReLU},
which returns the maximum of a neuron output and zero. 
Thus, ReLU activation function can be expressed as $ReLU(x) = max(x, 0)$.

We develop a SMC Programming algorithm by taking advantage that ReLU is a piece-wise linear activation function.
In such an approach, a Boolean satisfiability solver assigns the phases of neurons, and a convex programming solver
determines if all real-valued constraints can be satisfied. 
Thus, after fixing ReLU phases, NN becomes an affine function and inputs to the NN need to satisfy the selected phases. 

In particular, we consider NN as a feedback controller that makes decision based on current states of robot and observation of environment:
\begin{equation}    
    \label{eq:nn_nonlinear}    
    u_t = f_{NN}(d_t)
\end{equation}    
For a certain choice of ReLU phases, the piecewise affine function represents NN \eqref{eq:nn_nonlinear} is equivalent to the following linear constraints:
\begin{equation}
    \label{eq:nn_linear}  
    u_t =  G_b d_t + h_b, \qquad Q_b d_t \le c_b
\end{equation}
where $G_b$, $h_b$, $Q_b$, $c_b$ depend on the selected ReLU phases and their dimensions are determined by the size of input image to NN.



\subsection{Problem Definition}

\begin{definition}
    \textit{(Input Problem Instance)}: An input problem instance is defined as the tuple 
    $\mathcal{P} = \langle \W, (A, B), \overline{u}, \overline{x}, LiDAR, NN, \varphi \rangle$, where:
    \begin{itemize}
        \item $\W$ is workspace,
        \item $(A, B)$ is the robot dynamics,
        \item $\overline{u}$ is the bound on the robot inputs,
        \item $\overline{x}$ is the bound on the robot states,
        \item $LiDAR$ is a LiDAR scanner that measures distances to obstacles in a set of specified directions,
        \item $NN$ is a trained neural network, 
        \item $\varphi$ is a system-level property captured by LTL
    \end{itemize}
\end{definition}    


\begin{definition}
    \textit{(Valid trajectory)}: For an input problem instance 
    $\mathcal{P} = \langle \W, (A, B), \overline{u}, \overline{x}, LiDAR, NN, \varphi \rangle$,
    a trajectory $\{x_t: t \in \N\}$, is called \textit{valid}, 
    if the constraints ~\eqref{eq:dyn}, ~\eqref{eq:ulimit}, ~\eqref{eq:distance}, ~\eqref{eq:image}, ~\eqref{eq:nn_nonlinear} hold.
\end{definition}

We now formally define system-level neural network verification problem that we solve in this paper:
\begin{definition}
    \textit{(System-Level Neural Network Verification Problem)}: Given an input problem instance 
    $\mathcal{P} = \langle \W, (A, B), \overline{u}, \overline{x}, LiDAR, NN, \varphi \rangle$,
    find a region such that all trajectories start from the region is both valid and satisfy $\varphi$.
\end{definition}    



\subsection{Linear Temporal Logic}

{\color{blue} LTL background.}




\section{Framework}

The system-level neural network verification problem appears to have two challenges:
ReLU activation function is nonlinear, and LiDAR image dependence on current position of robot is also nonlinear.
We tackle the first one by developing a SMC Programming algorithm,
and the second one by partitioning workspace into regions followed by building a state machine 
that captures finite transition abstraction.


\subsection{SMC Programming}

SMC Programming is designed to efficiently reason about Boolean and convex constraints 
at the same time~\cite{Shoukry2018SMC} by integrating SAT solving and convex optimization.
This approach is a natural choice in reasoning NN due to the piece-wise linearity of ReLU activation function.
Specifically, a Boolean satisfiability solver assigns the phases of neurons, and a convex programming solver 
determines if all real-valued constraints can be satisfied. 
For a certain selection of ReLU phases, NN becomes an affine function with limit on NN inputs to satisfy the selected phases. 
Thus, the nonlinear function represents NN \eqref{eq:nn_nonlinear} is equivalent to the following affine constraints:
\begin{equation}
    \label{eq:nn_linear}  
    u_t =  G_b d_t + h_b, \quad Q_b d_t \le c_b, \qquad \forall t \in \N 
\end{equation}
where $G_b$, $h_b$, $Q_b$, $c_b$ depend on the selection of ReLU phases, and have appropriate dimensions 
determined by NN and its input size.




\subsection{LiDAR Image Function}

In general, LiDAR image dependence on robot position is nonlinear even with workspace fixed.
It only becomes affine when robot is constrained in a small region such that 
the obstacle boundary intersects laser in each direction is fixed.
We formally define the collection of obstacles detected by a LiDAR, followed by claiming that LiDAR image is a piecewise affine function.

\begin{definition}
    A LiDAR configuration is a list of obstacle boundaries detected by a LiDAR, and the list is sorted by laser directions.
\end{definition}

\begin{theorem}
    For a given workspace with polyhedral obstacles, if laser directions are fixed, 
    and LiDAR detection range is large enough such that each laser intersects an obstacle boundary, 
    then LiDAR image is a piecewise affine function of robot position:
    \begin{equation}
        \label{eq:image_func}
        d_t = P_j [x_{p_t} \quad y_{p_t}]^\intercal + q_j, \quad \forall j \in {1, ..., M} 
    \end{equation}
    where $M$ is number of regions identified by LiDAR configuration.
\end{theorem}    

To prove this theorem, we propose an algorithm that partition workspace based on LiDAR configuration.

\begin{proof}
    First, partition free space by segments that have one end be obstacle vertices, one end on obstacle boundaries, 
    and oriented along laser directions.
    Such partition segments can be found by assuming rays start from obstacle vertices,
    in all laser directions except those not point to free space.
    Intersections between these rays with the first obstacle boundary encountered would be the other endpoint of partition segments.
    Next, all intersections between partition segments can be found by a sweep algorithm~\cite{CGbook}. 
    {\color{blue} More detail about sweep if necessary.}
    Consider a graph consists of all partition segments and intersections,
    then each region that cannot be further divided can be found by graph search.
    Each of these regions corresponds to a unique LiDAR configuration.
    
    Assume robot equipped with a LiDAR scanner is constrained in an arbitrary one of these regions, 
    then obstacle boundary intersects with each laser is fixed due to the uniqueness of LiDAR configuration for each region.
    In other words, for each laser direction $\theta^i \in {1, ..., N}$, there exists $0 \le k^i \le 1$ such that:
    \begin{equation}
        \label{eq: intersection1}
        x^i = x_{p_t} + k^i R cos\theta^i, \qquad y^i = y_{p_t} + k^i R sin\theta^i, \\
    \end{equation} 
    \begin{equation}
        \label{eq: intersection2}
        x_{obs}^s \le x^i \le x_{obs}^l, \qquad y_{obs}^s \le y^i \le y_{obs}^l
    \end{equation} 
    where constant $R$ is maximum detection range of LiDAR.
    Constants $x_{obs}^s$, $x_{obs}^l$, $y_{obs}^s$, $y_{obs}^l$ parameterize the obstacle boundary intersects laser in direction $\theta^i$.
    Since laser directions do not rotate when robot turn around,  
    $\theta^i$ are constants and hence $x^i$, $y^i$ are affine functions of robot position $(x_{p_t}$, $y_{p_t})$.
    Therefore, LiDAR image $d_t$ that consists of linear terms of $x^i$, $y^i$ is an affine function of robot position.
\end{proof}

As a special case, consider obstacles are all rectangles. 
A laser in direction $\theta^i$, where $i \in {1, ..., N}$, intersects a vertical obstacle 
defined by lower end $(x_{obs}, y_{obs}^s)$ and upper end $(x_{obs}, y_{obs}^l)$, 
constraints ~\eqref{eq: intersection1}, ~\eqref{eq: intersection2} can be simplified as:
\begin{equation}
    \label{eq:vertical}
    x^i = x_{obs}, \qquad y^i = y_{p_t} + (x_{obs} - x_{p_t}) tan \theta^i
\end{equation}
Similarly, for a horizontal obstacle boundary with left end $(x_{obs}^s, y_{obs})$ and right end $(x_{obs}^l, y_{obs})$,
constraints ~\eqref{eq: intersection1}, ~\eqref{eq: intersection2} can be written as:
\begin{equation} 
    \label{eq:horizontal}
    x^i = x_{p_t} + (y_{obs} - y_{p_t}) cot \theta^i, \qquad y^i = y_{obs}
\end{equation}


\begin{theorem}
    Given workspace with polyhedral obstacles, running time of partitioning workspace based on LiDAR configuration is
    $\mathcal{O}(n\log{}n + I\log{}n)$, where $n$ is number of partition line segments 
    and $I$ is number of intersection points of segments.
    \begin{proof}
        {\color{blue} This is complexity of sweep algorithm.} 
    \end{proof}
\end{theorem}    

{\color{blue} In practice have numerical errors.}







\subsection{State Machine}


For a given ReLU assignment, safety verification involves both feedback control loop and environment 
is equivalent to solve a feasibility problem with constraints 
~\eqref{eq:dyn}, ~\eqref{eq:ulimit}, ~\eqref{eq:nn_linear}, ~\eqref{eq:image_func},.
However, this optimization problem is non-convex due to the piecewise linearity of LiDAR image function.
Instead of solving the problem as a whole, we consider transition feasibility between 
each pair of regions, each corresponds to a fixed LiDAR configuration. 
Thus, we add constraints that quadrotor is located in region $R_j$ currently and moves to region $R_{j^\prime}$,
$\forall j \neq j^\prime \in {1, ..., M}$:
\begin{equation}
    \label{eq:region}
    E_j[x_{p_t} \quad y_{p_t}]^\intercal <= f_j, \qquad E_{j^\prime}[x_{p_{t+1}} \quad y_{p_{t+1}}]^\intercal <= f_{j^\prime}
\end{equation}
Based on transition feasibility between all regions, a state machine is available and can be used to verify
LTL specifications.

{\color{blue} How to use state machine to verify LTL?}
{\color{blue} Any limit on LTL formula?}

{\color{blue} Talk about abstraction refinement.}


\section{Algorithm Analysis}

\subsection{Hardness of Building State Machine}

\begin{theorem}
    Given two regions from the partitioned workspace, the problem of determining whether constraints
    ~\eqref{eq:dyn}, ~\eqref{eq:ulimit}, ~\eqref{eq:nn_nonlinear}, ~\eqref{eq:image}, ~\eqref{eq:image_func}
    ~\eqref{eq:region}
    can be satisfied simultaneously for a given neural network is NP-complete.
\end{theorem}    

\begin{proof}
    {\color{blue} 
    (The problem is NP) A certificate can be simply checked by forwarding through NN.

    (NP-hard) It can be reduced to the problem of determining whether input and output constraint of NN is satisfiable, 
    which has been proved to be NP-complete by Reluplex paper.
    }
\end{proof}


\begin{theorem}
    The problem of determining whether there exists two regions from the partitioned workspace such that constraints
    ~\eqref{eq:dyn}, ~\eqref{eq:ulimit}, ~\eqref{eq:nn_nonlinear}, ~\eqref{eq:image}, ~\eqref{eq:image_func}
    ~\eqref{eq:region}
    can be satisfied simultaneously for a given neural network is $\Delta_2^P = P^{NP}$.
\end{theorem}    

\begin{proof}
    {\color{blue} 
    The problem can be solved in polynomial time by a Turing machine with an oracle for NP-complete problem, 
    i.e. polynomially many calls of NP-complete problem.
    }
\end{proof}    



{\color{blue} Are above two theorems enough or need complexity of building state machine, which is not a decision problem?}


\subsection{Soundness and Completeness}
{\color{blue} Sound but not complete.}


\input{text/results}



\bibliographystyle{ACM-Reference-Format}
\bibliography{sample-bibliography}

\end{document}
